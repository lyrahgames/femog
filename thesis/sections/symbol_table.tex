\documentclass[crop=false,9pt]{standalone}
\usepackage{standard}
\usepackage{tabularx}

\begin{document}

	\section*{Symboltabelle}
	% \setlength\extrarowheight{5pt}
	\renewcommand{\arraystretch}{1.3}
	\begin{table}[H]
		\begin{tabularx}{\textwidth}{p{0.15\textwidth}p{0.84\textwidth}}
			\hline
			\textbf{Symbol} & \textbf{Definition} \\
			\hline
			\hline \\

			$\exists!\ldots :\ldots$ & Es existiert genau ein $\ldots$, sodass $\ldots$ gilt. \\

			$x\in A$ & $x$ ist ein Element der Menge $A$. \\

			$A\subset B$ & $A$ ist eine Teilmenge von $B$. \\

			$A\cap B$ & $\set[x\in B]{x\in A}$ für Mengen $A,B$ --- Mengenschnitt \\

			$A\cup B$ & $\set[x\in A \text{ oder } x\in B]{x}$ für Mengen $A,B$ --- Mengenvereinigung \\

			$A\setminus B$ & $\set[x\not\in B]{x\in A}$ für Mengen $A,B$ --- Differenzmenge \\

			$A\times B$ & $\set[x\in A,y\in B]{(x,y)}$ für Mengen $A$ und $B$ --- kartesisches Produkt \\

			$\emptyset$ & leere Menge \\

		\end{tabularx}
	\end{table}

\end{document}

