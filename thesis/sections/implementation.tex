\documentclass[crop=false,10pt,ngerman]{standalone}
\usepackage{standard}
\begin{document}
  \section{Implementierung} % (fold)
  \label{sec:implementierung}

    \subsection{Repräsentation des Berechnungsgebietes} % (fold)
    \label{sub:repräsentation_des_berechnungsgebietes}
      Das Gebiet $\Omega$ kann nur in Dreiecke unterteilt werden.
      Wir trennen Eckpunkte und Primitive, wie im OBJ-Dateiformat.
      Weiterhin werden Kanten für Neumann- und Dirichlet-Randbedingungen extra aufgeführt.
      Dies ist möglich, da die Randbedingungen unabhängig von den Primitiven betrachtet werden können.
      Die Primitive müssen demnach nicht auf die Ränder verweisen.
      \cite{Alberty1998}
    % subsection repräsentation_des_berechnungsgebietes (end)

    \subsection{Konstruktion der Mass und Stiffness Matrix} % (fold)
    \label{sub:konstruktion_der_mass_und_stiffness_matrix}

    % subsection konstruktion_der_mass_und_stiffness_matrix (end)

    \subsection{Konstruktion des linearen Gleichungssystems} % (fold)
    \label{sub:konstruktion_des_linearen_gleichungssystems}

    % subsection konstruktion_des_linearen_gleichungssystems (end)

    \subsection{Lösen des Gleichungssystems} % (fold)
    \label{sub:lösen_des_gleichungssystems}

    % subsection lösen_des_gleichungssystems (end)

    \subsection{Randbedingungen} % (fold)
    \label{sub:randbedingungen}

    % subsection randbedingungen (end)

    \subsection{Visualisierung} % (fold)
    \label{sub:visualisierung}

    % subsection visualisierung (end)
  % section implementierung (end)
\end{document}