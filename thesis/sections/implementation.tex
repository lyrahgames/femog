\documentclass[crop=false,10pt,ngerman]{standalone}
\usepackage{standard}
\begin{document}
  \section{Implementierung} % (fold)
  \label{sec:implementierung}

    \subsection{Repräsentation der Daten} % (fold)
    \label{sub:repräsentation_der_daten}
      Das Gebiet $\Omega$ kann nur in Dreiecke unterteilt werden.
      Wir trennen Eckpunkte und Primitive, wie im OBJ-Dateiformat.
      Weiterhin werden Kanten für Neumann- und Dirichlet-Randbedingungen extra aufgeführt.
      Dies ist möglich, da die Randbedingungen unabhängig von den Primitiven betrachtet werden können.
      Die Primitive müssen demnach nicht auf die Ränder verweisen.
      \cite{Alberty1998}
    % subsection repräsentation_der_daten (end)

  % section implementierung (end)
\end{document}