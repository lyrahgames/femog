\section{Wärmeleitungsgleichung} % (fold)
\label{ssub:heat-equation}
  Die einfachste zeitabhängige partielle Differentialgleichung ist die Wärmeleitungsgleichung \cite[S.~175]{Schweizer2013}.
  Bei ihr handelt es sich um eine parabolische partielle Differentialgleichung zweiter Ordnung, die ein einfaches Modell für die Berechnung instationärer Temperaturfelder bildet.
  Physikalisch basiert sie auf dem Energieerhaltungssatz.

  Für die Einarbeitung eines Zeitparameters soll zunächst wieder ein räumliches Berechnungsgebiet gewählt werden, welches sich im Laufe der Zeit nicht verändert.
  Auf diesem lassen sich die räumlichen Ableitungen zu allen Zeiten analog zur Poisson-Gleichung betrachten.
  Um die Eindeutigkeit der Lösung auch nach dem Hinzufügen der Zeitableitungen zu gewährleisten, sind Anfangswerte von Nöten.
  Mit dieser Betrachtung lässt sich die Wärmeleitungsgleichung nun ohne neue Erkenntnisse analog zur Poisson-Gleichung definieren.
  \cite{Schweizer2013,Alberty1998}.

  \begin{definition}[Wärmeleitungsgleichung]
    Es seien $\boxBrackets{\Omega} \define \roundBrackets{\domain,\dirichletBoundary, \neumannBoundary, ν}$ ein Berechnungsgebiet und die folgenden Funktionen gegeben.
    \begin{align*}
      f&\in\setIntegrable^2(\domain\times[0,\infty))
      &
      u_\mathrm{D} &\in \setSobolev^1(\domain\times[0,\infty))
      \\
      u_0 &\in \setSobolev^1(\domain)
      &
      u_\mathrm{N} &\in \setIntegrable^2\roundBrackets{\neumannBoundary \times [0,\infty)}
    \end{align*}
    Eine Funktion $u\in\setSobolev^1(\domain\times[0,\infty))$ nennt man eine schwache Lösung der Wärmeleitungsgleichung, wenn die folgenden Gleichungen für alle $t\in[0,\infty)$ gelten.
    \begin{align*}
      \partial_t u - \laplacian u &= f & \text{(Wärmeleitungsgleichung)} \\
      u(\cdot,0) &= u_0 & \text{(Anfangsbedingungen)}\\
      u(\cdot,t)\vert_{\dirichletBoundary} &= u_\mathrm{D}(\cdot,t)\vert_{\dirichletBoundary} & \text{(Dirichlet-Randbedingungen)}\\
      \scalarProduct{\nabla u(\cdot,t)\vert_{\neumannBoundary}}{ν} &= u_\mathrm{N}(\cdot,t) & \text{(Neumann-Randbedingungen)}
    \end{align*}
    Das Tupel $(\boxBrackets{\Omega},f,u_0,u_\mathrm{D}, u_\mathrm{N},u)$ wird dann auch Wärmeleitungsproblem genannt.
  \end{definition}

  Ebenfalls analog zur Poisson-Gleichung ist es auch möglich, die Wärmeleitungsgleichung in eine schwache Formulierung zu transformieren.
  Auch bei diesem Vorgang entstehen keine Überraschungen bei der mathematischen Vorgehensweise.
  Zu bemerken sei hier, dass für die Konstruktion der Finite-Elemente-Methode die rigorose schwache Formulierung der Wärmeleitungsgleichung keine direkte Rolle spielt.
  Beim Übergang zur Diskretisierung erweist es sich als praktisches Vorgehen, jegliche Zeitableitung zuvor durch einen diskreten Differenzenquotienten zu ersetzen, sodass es sich bei dem resultierenden Gleichungssystem streng genommen nicht mehr um eine zeitabhängige Differentialgleichung handelt.
  \cite{Schweizer2013,Alberty1998}

  Für die zeitliche Diskretisierung der Wärmeleitungsgleichung kann man die sogenannte Rothe-Methode, wie sie in \cite[S.~211~ff]{Schweizer2013} und \cite{Alberty1998} beschrieben ist, verwenden.
  Sie basiert auf dem impliziten Euler-Verfahren und verhindert damit die Divergenz numerischer Lösungen \cite[S.~472~f]{Quarteroni2000}.
  Auch andere Zeitschrittverfahren, wie die Crank-Nicolson-Methode oder die Familie der Runge-Kutta-Verfahren sind jedoch denkbar \cite{Quarteroni2000,Cheney2008}.

  Es seien nun ein kleiner Zeitschritt $\infinitesimal{t}\in (0,\infty)$, ein Berechnungsgebiet $\boxBrackets{\domain} \define (\domain,\dirichletBoundary,\neumannBoundary,ν)$ und ein Wärmeleitungsproblem $(\boxBrackets{\domain},f,u_0,u^{(\mathrm{D})},u^{(\mathrm{N})},u)$ gegeben.
  Durch $\infinitesimal{t}$ wird das Zeitintervall $T\define[0,\infty)$ diskretisiert und durch die Menge $\boxBrackets{T}\define\set{k\cdot\infinitesimal{t}}{k\in\setNatural_0}$ dargestellt.
  Weiterhin bezeichne eine natürliche Zahl $n\in\setNatural$ als Index der zeitabhängigen Funktionen $f$, $u^{(\mathrm{D})}$, $u^{(\mathrm{N})}$ und $u$ deren Diskretisierung zum Zeitpunkt $n\cdot\infinitesimal{t}$.
  Dieses Schema soll am Beispiel von $u$ demonstriert werden.
  \[
    u \longleftrightarrow (u_n)_{n\in\setNatural_0}
    \separate
    u(\cdot,n\infinitesimal{t}) \longleftrightarrow u_n
  \]
  Zu beachten ist hier, dass die Diskretisierung $u_n$ im Allgemeinen $u(\cdot,n\infinitesimal{t})$ nur annähert und nicht exakt beschreibt.
  Für die Diskretisierung der zeitlichen Ableitung folgt durch die Anwendung des impliziten Euler-Verfahrens ein ähnliches Schema.
  \[
    \partial_t u \longleftrightarrow (\partial_t u_n)_{n\in\setNatural}
    \separate
    \partial_t u(\cdot,n\infinitesimal{t}) \longleftrightarrow \frac{u_{n} - u_{n-1}}{\infinitesimal{t}}
  \]
  Mit diesen Transformationen ist es jetzt möglich, die gesamte Wärmeleitungsgleichung zu diskretisieren, wie im folgenden Schema gezeigt.
  \[
    \partial_t u -\laplacian u = f
    \qquad \longleftrightarrow \qquad
    \frac{u_{n} - u_{n-1}}{\infinitesimal{t}} - \laplacian u_n = f_n
  \]
  Bei der Berechnung wird davon ausgegangen, dass $u_{n-1}$ entweder als Anfangswert $u_0$ gegeben ist oder durch einen vorherigen Zeitschritt berechnet wurde.
  Die einzige Unbekannte in der diskretisierten Gleichung ist damit $u_{n}$, für die man den folgenden Ausdruck erhält.
  \[
    \roundBrackets{\identity - \infinitesimal{t}\laplacian}u_{n} = \infinitesimal{t} f_{n} + u_{n-1}
  \]
  Für jeden Zeitpunkt $t\in\boxBrackets{T}$ handelt es sich hier um eine zeitunabhängige partielle Differentialgleichung zweiter Ordnung, deren schwache Lösung analog zu der der Poisson-Gleichung konstruiert werden kann.
  Zunächst werden die Dirichlet-Randbedingungen wieder über eine Transformation in die Gleichung eingearbeitet.
  \[
    s_n\in\setSobolev^1_\mathrm{D}(\domain)
    \separate
    s_n\define u_n - u_n^{(\mathrm{D})}
  \]
  Durch die Integration mit Testfunktionen und die Anwendung des Gaußschen Satzes erhält man nun für die schwachen Lösungen der Zeit-diskretisierten Wärmeleitungsgleichung die folgende etwas längliche Formulierung für alle $φ\in\setSobolev^1_\mathrm{D}(\domain)$.
  \begin{align*}
    &\integral{\domain}{}{s_nφ}{λ}
    + \infinitesimal{t}\integral{\domain}{}{\scalarProduct{\nabla s_n}{\nabla φ}}{λ} \\
    &= \infinitesimal{t}\boxBrackets{
      \integral{\domain}{}{f_n φ}{λ}
      + \integral{\neumannBoundary}{}{u_n^{(\mathrm{N})}φ}{σ}
      - \integral{\domain}{}{\scalarProduct{\nabla u_n^{(\mathrm{D)}}}{\nabla φ}}{λ}
    }
    \\
    &+ \integral{\domain}{}{u_{n-1} φ}{λ}
    - \integral{\domain}{}{u^{(\mathrm{D})}_{n}φ}{λ}
  \end{align*}
% subsubsection wärmeleitungsgleichung (end)