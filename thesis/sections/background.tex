\documentclass[crop=false]{standalone}
\usepackage{standard}
\begin{document}
  \section{Grundlagen} % (fold)
  \label{sec:background}

    \subsection{Modelprobleme} % (fold)
    \label{sub:modelprobleme}
      \newcommand{\domain}{\ensuremath{\Omega}}
      \newcommand{\boundary}{\ensuremath{\partial\domain}}
      \newcommand{\neumannBoundary}{\ensuremath{\boundary_\mathrm{N}}}
      \newcommand{\dirichletBoundary}{\ensuremath{\boundary_\mathrm{D}}}

      Für die Implementierung einer Finite-Elemente-Methode benötigt man zunächst ein mathematisches Modell, welches durch ein System partieller Differentialgleichungen beschrieben wird.
      Um sich jedoch nicht auf die Anwendung physikalischer Gesetze und auf das Verstehen in den darin resultierenden Modellen fokussieren zu müssen, werden in dieser Arbeit nur die einfachsten physikalisch basierten partiellen Differentialgleichungen als Grundlage vorausgesetzt.
      Die Poisson-, Wellen- und Wärmeleitungsgleichung beschreiben grundlegende Phänomene in der Elektrodynamik, Mechanik und Thermodynamik.
      Bei diesen drei Gleichungen handelt sich um eine elliptische, eine hyperbolische und eine parabolische lineare partielle Differentialgleichung zweiter Ordnung in simpelster Form.
      Gerade deswegen kann man durch diese Gleichungen Modelprobleme formulieren, an denen das Vorgehen der Finite-Elemente-Methode unkompliziert erläutert werden kann, die aber auch komplex genug sind, um das Verfahren auch auf schwierigere Modelle anwenden zu können.

      Um die Visualisierung und die Konstruktion von Testgebieten zu erleichtern, werden die Gleichungen nur mit zwei Raumdimensionen behaftet.
      Dies ändert nichts Grundlegendes an der Implementierung und dient zudem dem besseren Verständnis des Lesers.

      \subsubsection{Poisson-Gleichung} % (fold)
      \label{ssub:poisson_gleichung}
        Die Poisson-Gleichung stellt eine wichtige Grundlage für die Wellen- und Wärmeleitungsgleichung dar.
        Sie ist eine zeitunabhängige, elliptische lineare partielle Differentialgleichung zweiter Ordnung und gerade aus diesem Grund als ein Test für die Implementierung geeignet.

        Es sei $\domain\subset\setReal^2$ ein beschränktes Lipschitz-Gebiet mit einem polygonalen Rand $\boundary$.
        Die folgende Funktion ν beschreibe fast-überall die äußere Normale von $\domain$.
        \[
          \function{ν}{\boundary}{\setReal^2}
        \]
        Weiterhin sei $\dirichletBoundary$ eine abgeschlossene Teilmenge des Randes $\boundary$ mit positiver Länge und $\neumannBoundary$ deren Komplement.
        \[
          \boundary_{\mathrm{D}} \subset \boundary
          \separate
          σ\roundBrackets{\dirichletBoundary} > 0
          \separate
          \neumannBoundary \define \boundary \setminus \dirichletBoundary
        \]
        Auf der Menge $\dirichletBoundary$ sollen Dirichlet-Randbedingungen und auf deren Komplement $\neumannBoundary$ Neumann-Randbedingungen gefordert werden.
        Gegeben seien zudem die folgenden Funktionen.
        \[
          f\in\setIntegrable^2(\domain)
          \separate
          u_\mathrm{D} \in \setSobolev^1(\domain)
          \separate
          u_\mathrm{N} \in \setIntegrable^2\roundBrackets{\neumannBoundary}
        \]
        Die Funktion $f$ ist auch bekannt als die Volumenkraft.
        Die Funktionen $u_\mathrm{D}$ und $u_\mathrm{N}$ stellen die Dirichlet- und Neumann-Randbedingungen dar.
        Gesucht ist nun eine Funktion $u\in\setSobolev^1(\domain)$, sodass die folgenden Gleichungen erfüllt sind.
        \begin{align*}
          -\laplacian u &= f \\
          u\vert_{\dirichletBoundary} &= u_\mathrm{D}\vert_{\dirichletBoundary} \\
          \scalarProduct{\nabla u\vert_{\neumannBoundary}}{ν} &= u_\mathrm{N}
        \end{align*}
        % Zu beachten ist, dass die Gleichungen im Sinne der punktweisen Gleichheit nur fast-überall erfüllt sein müssen.
        In diesem Falle stellt $u$ eine Lösung der Poisson-Gleichung dar.
        Nach dem Lemma von Lax und Milgram ist bekannt, dass für das gegebene Poisson-Problem eine schwache Lösung existiert.
        Aus diesem Grund bilden wir zunächst die schwache Formulierung des Poisson-Problems.
        Wir arbeiten die inhomogenen Dirichlet-Randbedingungen durch die folgende Zerlegung in das Problem ein.
        \[
          v\in\setSobolev^1_\mathrm{D}(\domain) \define \set{w\in\setSobolev^1(\domain)}{w\vert_{\dirichletBoundary} = 0}
        \]
        \[
          v\define u - u_\mathrm{D}
        \]
        Die schwache Formulierung ergibt sich dann zu der folgenden Form.
        Für alle $w\in\setSobolev^1_\mathrm{D}(\domain)$ muss das folgende gelten.
        \[
          \integral{\domain}{}{\scalarProduct{\nabla u}{\nabla w}}{λ} = \integral{\domain}{}{fw}{λ} + \integral{\neumannBoundary}{}{u_\mathrm{N}w}{σ} - \integral{\domain}{}{\scalarProduct{\nabla u_\mathrm{D}}{\nabla w}}{λ}
        \]
      % subsubsection poisson_gleichung (end)

      \subsubsection{Wärmeleitungsgleichung} % (fold)
      \label{ssub:wärmeleitungsgleichung}
        Die einfachste zeitabhängige PDE ist die Wärmeleitungsgleichung.

        \begin{align*}
          \partial_t u - \laplacian u &= f \\
          u(\cdot,t_0) &= u_0 \\
          u(\cdot,t)\vert_{\dirichletBoundary} &= u_\mathrm{D}\vert_{\dirichletBoundary} \\
          \scalarProduct{\nabla u(\cdot,t)\vert_{\neumannBoundary}}{ν} &= u_\mathrm{N}
        \end{align*}
      % subsubsection wärmeleitungsgleichung (end)

      \subsubsection{Wellengleichung} % (fold)
      \label{ssub:wellengleichung}
        \begin{align*}
          \partial^2_t u - \laplacian u &= f \\
          u(\cdot,t_0) &= u_0 \\
          \partial_t u(\cdot,t_0) &= u_1 \\
          u(\cdot,t)\vert_{\dirichletBoundary} &= u_\mathrm{D}\vert_{\dirichletBoundary} \\
          \scalarProduct{\nabla u(\cdot,t)\vert_{\neumannBoundary}}{ν} &= u_\mathrm{N}
        \end{align*}
      % subsubsection wellengleichung (end)

    % subsection modelprobleme (end)

    \subsection{Finite Elemente Methode} % (fold)
    \label{sub:finite_elemente_methode}
      Die Finite-Elemente-Methode ist, wie bereits in Abschnitt \ref{sec:introduction} erwähnt, ein allgemeines numerisches Verfahren, um partielle Differentialgleichungen approximativ zu lösen.
      Genauer gesagt, handelt es sich um eine spezielle Form des Ritz-Galerkin-Verfahrens.
      Die Finite-Elemente-Formulierung eines Problems resultiert in einem System von algebraischen Gleichungen.
      Um ein Problem zu lösen, unterteilt es das Berechnungsgebiet in kleinere und einfach zu behandelnde Teile (engl.: \textit{primitives}), die auch finite Elemente genannt werden.
      Die einfachen Systeme von algebraischen Gleichungen der einzelnen finiten Elemente werden dann in ein großes System von Gleichungen, welches das gesamte Problem modelliert, assembliert.
      Durch die Minimierung einer Fehlerfunktion mithilfe verschiedener Variationsmethoden ist die Finite-Elemente-Methode dann in der Lage, das Problem zu lösen.

      \begin{itemize}
        \item schwache formulierung
        \item diskretisierung des gebiets
        \item wahl der basisfunktionen (diskretisierung des funktionenraums) mit kleinem träger
        \item matrixform des problems
        \item lösen des matrixproblems
      \end{itemize}
    % subsection finite_elemente_methode (end)

    \subsection{Sparse Matrix Methoden} % (fold)
    \label{sub:sparse_matrix_methoden}

    % subsection sparse_matrix_methoden (end)

    \subsection{Aufbau und Funktionsweise einer Grafikkarte} % (fold)
    \label{sub:aufbau_und_funktionsweise_einer_grafikkarte}

    % subsection aufbau_und_funktionsweise_einer_grafikkarte (end)

  % section grundlagen (end)
\end{document}