\documentclass[crop=false]{standalone}
\usepackage{standard}
\begin{document}
  \section{Grundlagen} % (fold)
  \label{sec:background}


    \subsection{Finite Elemente Methode} % (fold)
    \label{sub:finite_elemente_methode}

    % subsection finite_elemente_methode (end)

    \subsection{Modelprobleme} % (fold)
    \label{sub:modelprobleme}
      \newcommand{\domain}{\ensuremath{\Omega}}
      \newcommand{\boundary}{\ensuremath{\partial\domain}}
      \newcommand{\neumannBoundary}{\ensuremath{\boundary_\mathrm{N}}}
      \newcommand{\dirichletBoundary}{\ensuremath{\boundary_\mathrm{D}}}

      \subsubsection{Poisson-Gleichung} % (fold)
      \label{ssub:poisson_gleichung}
        Die Poisson-Gleichung stellt eine wichtige Grundlage dar.
        Sie ist zeitunabhängig, elliptisch und ein grundlegender Bestandteil typischer zeitabhängiger PDE, wie zum Beispiel der Wärmeleitungs- und Wellengleichung.
        Sie stellt ein ideales Testproblem dar, um zu überprüfen, ob die implementierten Verfahren funktionieren.

        Es sei $\domain\subset\setReal^2$ ein beschränktes Lipschitz-Gebiet mit einem polygonalen Rand $\boundary$.
        Die folgende Funktion ν beschreibe fast-überall die äußere Normale von $\domain$.
        \[
          \function{ν}{\boundary}{\setReal^2}
        \]
        Weiterhin sei $\dirichletBoundary$ eine abgeschlossene Teilmenge des Randes $\boundary$ mit positiver Länge und $\neumannBoundary$ deren Komplement.
        \[
          \boundary_{\mathrm{D}} \subset \boundary
          \separate
          σ\roundBrackets{\dirichletBoundary} > 0
          \separate
          \neumannBoundary \define \boundary \setminus \dirichletBoundary
        \]
        Auf der Menge $\dirichletBoundary$ sollen Dirichlet-Randbedingungen und auf deren Komplement $\neumannBoundary$ Neumann-Randbedingungen gefordert werden.
        Gegeben seien zudem die folgenden Funktionen.
        \[
          f\in\setIntegrable^2(\domain)
          \separate
          u_\mathrm{D} \in \setSobolev^1(\domain)
          \separate
          u_\mathrm{N} \in \setIntegrable^2\roundBrackets{\neumannBoundary}
        \]
        Die Funktion $f$ ist auch bekannt als die Volumenkraft.
        Die Funktionen $u_\mathrm{D}$ und $u_\mathrm{N}$ stellen die Dirichlet- und Neumann-Randbedingungen dar.
        Gesucht ist nun eine Funktion $u\in\setSobolev^1(\domain)$, sodass die folgenden Gleichungen erfüllt sind.
        \begin{align*}
          -\laplacian u &= f \\
          u\vert_{\dirichletBoundary} &= u_\mathrm{D}\vert_{\dirichletBoundary} \\
          \scalarProduct{\nabla u\vert_{\neumannBoundary}}{ν} &= u_\mathrm{N}
        \end{align*}
        % Zu beachten ist, dass die Gleichungen im Sinne der punktweisen Gleichheit nur fast-überall erfüllt sein müssen.
        In diesem Falle stellt $u$ eine Lösung der Poisson-Gleichung dar.
        Nach dem Lemma von Lax und Milgram ist bekannt, dass für das gegebene Poisson-Problem eine schwache Lösung existiert.
        Aus diesem Grund bilden wir zunächst die schwache Formulierung des Poisson-Problems.
        Wir arbeiten die inhomogenen Dirichlet-Randbedingungen durch die folgende Zerlegung in das Problem ein.
        \[
          v\in\setSobolev^1_\mathrm{D}(\domain) \define \set{w\in\setSobolev^1(\domain)}{w\vert_{\dirichletBoundary} = 0}
        \]
        \[
          v\define u - u_\mathrm{D}
        \]
        Die schwache Formulierung ergibt sich dann zu der folgenden Form.
        Für alle $w\in\setSobolev^1_\mathrm{D}(\domain)$ muss das folgende gelten.
        \[
          \integral{\domain}{}{\scalarProduct{\nabla u}{\nabla w}}{λ} = \integral{\domain}{}{fw}{λ} + \integral{\neumannBoundary}{}{u_\mathrm{N}w}{σ} - \integral{\domain}{}{\scalarProduct{\nabla u_\mathrm{D}}{\nabla w}}{λ}
        \]
      % subsubsection poisson_gleichung (end)

      \subsubsection{Wärmeleitungsgleichung} % (fold)
      \label{ssub:wärmeleitungsgleichung}
        Die einfachste zeitabhängige PDE ist die Wärmeleitungsgleichung.

        \begin{align*}
          \partial_t u - \laplacian u &= f \\
          u(\cdot,t_0) &= u_0 \\
          u(\cdot,t)\vert_{\dirichletBoundary} &= u_\mathrm{D}\vert_{\dirichletBoundary} \\
          \scalarProduct{\nabla u(\cdot,t)\vert_{\neumannBoundary}}{ν} &= u_\mathrm{N}
        \end{align*}
      % subsubsection wärmeleitungsgleichung (end)

      \subsubsection{Wellengleichung} % (fold)
      \label{ssub:wellengleichung}
        \begin{align*}
          \partial^2_t u - \laplacian u &= f \\
          u(\cdot,t_0) &= u_0 \\
          \partial_t u(\cdot,t_0) &= u_1 \\
          u(\cdot,t)\vert_{\dirichletBoundary} &= u_\mathrm{D}\vert_{\dirichletBoundary} \\
          \scalarProduct{\nabla u(\cdot,t)\vert_{\neumannBoundary}}{ν} &= u_\mathrm{N}
        \end{align*}
      % subsubsection wellengleichung (end)

    % subsection modelprobleme (end)

    \subsection{Galerkin Methode} % (fold)
    \label{sub:galerkin_methode}

    % subsection galerkin_methode (end)

    \subsection{Sparse Matrix Methoden} % (fold)
    \label{sub:sparse_matrix_methoden}

    % subsection sparse_matrix_methoden (end)

    \subsection{Aufbau und Funktionsweise einer Grafikkarte} % (fold)
    \label{sub:aufbau_und_funktionsweise_einer_grafikkarte}

    % subsection aufbau_und_funktionsweise_einer_grafikkarte (end)

  % section grundlagen (end)
\end{document}