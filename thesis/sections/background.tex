\documentclass[crop=false]{standalone}
\usepackage{standard}
\begin{document}
  \section{Grundlagen} % (fold)
  \label{sec:background}

    \subsection{Modelprobleme} % (fold)
    \label{sub:modelprobleme}
      \newcommand{\domain}{\ensuremath{\Omega}}
      \newcommand{\boundary}{\ensuremath{\partial\domain}}
      \newcommand{\neumannBoundary}{\ensuremath{\boundary_\mathrm{N}}}
      \newcommand{\dirichletBoundary}{\ensuremath{\boundary_\mathrm{D}}}

      Für die Implementierung einer Finite-Elemente-Methode benötigt man zunächst ein mathematisches Modell, welches durch ein System partieller Differentialgleichungen beschrieben wird.
      Um sich jedoch nicht auf die Anwendung physikalischer Gesetze und auf das Verstehen in den darin resultierenden Modellen fokussieren zu müssen, werden in dieser Arbeit nur die einfachsten physikalisch basierten partiellen Differentialgleichungen als Grundlage vorausgesetzt.
      Die Poisson-, Wellen- und Wärmeleitungsgleichung beschreiben grundlegende Phänomene in der Elektrodynamik, Mechanik und Thermodynamik.
      Bei diesen drei Gleichungen handelt sich um eine elliptische, eine hyperbolische und eine parabolische lineare partielle Differentialgleichung zweiter Ordnung in simpelster Form.
      Gerade deswegen kann man durch diese Gleichungen Modelprobleme formulieren, an denen das Vorgehen der Finite-Elemente-Methode unkompliziert erläutert werden kann, die aber auch komplex genug sind, um das Verfahren auch auf schwierigere Modelle anwenden zu können.

      Um die Visualisierung und die Konstruktion von Testgebieten zu erleichtern, werden die Gleichungen nur mit zwei Raumdimensionen behaftet.
      Dies ändert nichts Grundlegendes an der Implementierung und dient zudem dem besseren Verständnis des Lesers.

      \subsubsection{Berechnungsgebiet} % (fold)
      \label{ssub:berechnungsgebiet}
        Es sei $\domain\subset\setReal^2$ ein beschränktes Lipschitz-Gebiet mit einem polygonalen Rand $\boundary$.
        Die folgende Funktion ν beschreibe fast-überall die äußere Normale von $\domain$.
        \[
          \function{ν}{\boundary}{\setReal^2}
        \]
        Weiterhin sei $\dirichletBoundary$ eine abgeschlossene Teilmenge des Randes $\boundary$ mit positiver Länge und $\neumannBoundary$ deren Komplement.
        \[
          \boundary_{\mathrm{D}} \subset \boundary
          \separate
          σ\roundBrackets{\dirichletBoundary} > 0
          \separate
          \neumannBoundary \define \boundary \setminus \dirichletBoundary
        \]
        Auf der Menge $\dirichletBoundary$ sollen Dirichlet-Randbedingungen und auf deren Komplement $\neumannBoundary$ Neumann-Randbedingungen gefordert werden.
        Abbildung \ref{fig:domain} zeigt diese Definition an einem Beispiel.
        \begin{figure}[h]
          \center
          \includegraphics[width=0.5\textwidth]{images/domain.pdf}
          \caption{Die Abbildung zeigt ein Beispiel für ein Berechnungsgebiet.}
          \label{fig:domain}
        \end{figure}
      % subsection berechnungsgebiet (end)

      \subsubsection{Poisson-Gleichung} % (fold)
      \label{ssub:poisson_gleichung}
        Die Poisson-Gleichung stellt eine wichtige Grundlage für die Wellen- und Wärmeleitungsgleichung dar.
        Sie ist eine zeitunabhängige, elliptische lineare partielle Differentialgleichung zweiter Ordnung und gerade aus diesem Grund als ein Test für die Implementierung geeignet.

        Gegeben seien zudem die folgenden Funktionen.
        \[
          f\in\setIntegrable^2(\domain)
          \separate
          u_\mathrm{D} \in \setSobolev^1(\domain)
          \separate
          u_\mathrm{N} \in \setIntegrable^2\roundBrackets{\neumannBoundary}
        \]
        Die Funktion $f$ ist auch bekannt als die Volumenkraft.
        Die Funktionen $u_\mathrm{D}$ und $u_\mathrm{N}$ stellen die Dirichlet- und Neumann-Randbedingungen dar.
        Gesucht ist nun eine Funktion $u\in\setSobolev^1(\domain)$, sodass die folgenden Gleichungen erfüllt sind.
        \begin{align*}
          -\laplacian u &= f \\
          u\vert_{\dirichletBoundary} &= u_\mathrm{D}\vert_{\dirichletBoundary} \\
          \scalarProduct{\nabla u\vert_{\neumannBoundary}}{ν} &= u_\mathrm{N}
        \end{align*}
        % Zu beachten ist, dass die Gleichungen im Sinne der punktweisen Gleichheit nur fast-überall erfüllt sein müssen.
        In diesem Falle stellt $u$ eine Lösung der Poisson-Gleichung dar.
        Nach dem Lemma von Lax und Milgram ist bekannt, dass für das gegebene Poisson-Problem eine schwache Lösung existiert.
        Aus diesem Grund bilden wir zunächst die schwache Formulierung des Poisson-Problems.
        Wir arbeiten die inhomogenen Dirichlet-Randbedingungen durch die folgende Zerlegung in das Problem ein.
        \[
          v\in\setSobolev^1_\mathrm{D}(\domain) \define \set{w\in\setSobolev^1(\domain)}{w\vert_{\dirichletBoundary} = 0}
        \]
        \[
          v\define u - u_\mathrm{D}
        \]
        Die schwache Formulierung ergibt sich dann zu der folgenden Form.
        Für alle $w\in\setSobolev^1_\mathrm{D}(\domain)$ muss das folgende gelten.
        \[
          \integral{\domain}{}{\scalarProduct{\nabla u}{\nabla w}}{λ} = \integral{\domain}{}{fw}{λ} + \integral{\neumannBoundary}{}{u_\mathrm{N}w}{σ} - \integral{\domain}{}{\scalarProduct{\nabla u_\mathrm{D}}{\nabla w}}{λ}
        \]
      % subsubsection poisson_gleichung (end)

      \subsubsection{Wärmeleitungsgleichung} % (fold)
      \label{ssub:wärmeleitungsgleichung}
        Die einfachste zeitabhängige PDE ist die Wärmeleitungsgleichung.

        \begin{align*}
          \partial_t u - \laplacian u &= f \\
          u(\cdot,t_0) &= u_0 \\
          u(\cdot,t)\vert_{\dirichletBoundary} &= u_\mathrm{D}\vert_{\dirichletBoundary} \\
          \scalarProduct{\nabla u(\cdot,t)\vert_{\neumannBoundary}}{ν} &= u_\mathrm{N}
        \end{align*}
      % subsubsection wärmeleitungsgleichung (end)

      \subsubsection{Wellengleichung} % (fold)
      \label{ssub:wellengleichung}
        \begin{align*}
          \partial^2_t u - \laplacian u &= f \\
          u(\cdot,t_0) &= u_0 \\
          \partial_t u(\cdot,t_0) &= u_1 \\
          u(\cdot,t)\vert_{\dirichletBoundary} &= u_\mathrm{D}\vert_{\dirichletBoundary} \\
          \scalarProduct{\nabla u(\cdot,t)\vert_{\neumannBoundary}}{ν} &= u_\mathrm{N}
        \end{align*}
      % subsubsection wellengleichung (end)

    % subsection modelprobleme (end)

    \subsection{Finite-Elemente-Methode} % (fold)
    \label{sub:finite_elemente_methode}
      Die Finite-Elemente-Methode ist, wie bereits in Abschnitt \ref{sec:introduction} erwähnt, ein allgemeines numerisches Verfahren, um partielle Differentialgleichungen approximativ zu lösen.
      Genauer gesagt, handelt es sich um eine spezielle Form des Ritz-Galerkin-Verfahrens.
      Die Finite-Elemente-Formulierung eines Problems resultiert in einem System von algebraischen Gleichungen.
      Um ein Problem zu lösen, unterteilt es das Berechnungsgebiet in kleinere und einfach zu behandelnde Teile (engl.: \textit{primitives}), die auch finite Elemente genannt werden.
      Die einfachen Systeme von algebraischen Gleichungen der einzelnen finiten Elemente werden dann in ein großes System von Gleichungen, welches das gesamte Problem modelliert, assembliert.
      Durch die Minimierung einer Fehlerfunktion mithilfe verschiedener Variationsmethoden ist die Finite-Elemente-Methode dann in der Lage, das Problem zu lösen.
      Im Folgenden wird die grundlegende Vorgehensweise an den Modelproblemen illustriert.

      \paragraph{Schritt 1: Konstruktion der schwachen Formulierung} % (fold)
      \label{par:schritt_1_konstruktion_der_schwachen_formulierung}
        \hfill\\
        Bereits in der Theorie ist es sinnvoll, die Lösungen von partiellen Differentialgleichungen in einer abgeschwächten Form zu verstehen, da bereits nicht stetige Quellterme die Forderungen an eine klassische Lösung verletzen \cite[S.~46]{Schweizer2013}.
        Demzufolge scheint eine klassische Formulierung von partiellen Differentialgleichungen für die numerische Mathematik ungeeignet.
        Für die Definition einer schwachen Lösung beziehungsweise schwachen Formulierung verallgemeinert man die Ableitung einer Funktion mithilfe von Distributionen \cite[S.~46~ff]{Schweizer2013}.
        In den Sobolevräumen werden dann die integrierbaren Funktionen zusammengefasst, die im Sinne einer Distribution wieder eine integrierbare Ableitung besitzen \cite[S.~54~ff]{Schweizer2013}.
      % paragraph schritt_1_konstruktion_der_schwachen_formulierung (end)

      \paragraph{Schritt 2: Diskretisierung des Berechnungsgebietes} % (fold)
      \label{par:schritt_2_diskretisierung_des_berechnungsgebietes}
        \hfill\\
        Aufgrund der Funktionsweise eines Computers ist man bei numerischen Simulationen gezwungen, kontinuierliche Probleme auf die eine oder andere Art und Weise zu diskretisieren.
        In der Finite-Elemente-Methode findet diese Diskretisierung durch die Einführung von finiten Elementen statt.
        Das Berechnungsgebiet $\domain$ und dessen Rand $\boundary$ werden in ein äquivalentes System von mehreren finiten Elementen überführt.
        Dabei wird die Anzahl, der Typ und die Größe dieser Elemente durch den Modellierer festgelegt.
        Für zwei Raumdimensionen werden zumeist Dreiecke und Vierecke gewählt.
        \begin{figure}[h]
          \center
          \includegraphics[width=0.5\textwidth]{images/domain_one_dimension_example.pdf}
          \caption{}
          \label{fig:domain-example}
        \end{figure}
        \begin{figure}[h]
          \begin{subfigure}[b]{0.5\textwidth}
            \center
            \includegraphics[width=0.6\textwidth]{images/domain_discretization_1.pdf}
          \end{subfigure}
          \begin{subfigure}[b]{0.5\textwidth}
            \center
            \includegraphics[width=0.6\textwidth]{images/domain_discretization_2.pdf}
          \end{subfigure}
          \caption{%
            Die rechte Abbildung zeigt ein Beispiel für die Diskretisierung des polygonalen Berechnungsgebietes aus dem linken Bild durch die Verwendung von Dreiecken als finiten Elementen.%
          }
          \label{fig:domain-discretization}
        \end{figure}
      % paragraph schritt_2_diskretisierung_des_berechnungsgebietes (end)

      \paragraph{Schritt 3: Wahl der Basisfunktionen} % (fold)
      \label{par:schritt_3_wahl_der_basisfunktionen}
        \hfill\\
        Neben der Diskretisierung des Berechnungsgebietes $\domain$ ist auch die Diskretisierung der Funktionenräume $\setSobolev^1(\domain)$ und $\setSobolev^1_\mathrm{D}(\domain)$ nötig.
        Dies erreicht man durch die Wahl endlich vieler Basisfunktionen.
        Im Idealfall sollten diese einen kleinen Träger aufweisen, um die spätere Berechnung effizienter zu gestalten.
        Gerade für partielle Differentialgleichung zweiter Ordnung ist die Wahl von stückweise linearen Basisfunktionen ausreichend.
        \begin{figure}[h]
          \center
          \includegraphics[width=0.5\textwidth]{images/hat_function_one_dimension_example.pdf}
          \caption{}
          \label{fig:hat-function-example}
        \end{figure}
        \begin{figure}[h]
          \center
          \includegraphics[width=0.5\textwidth]{images/hat_function.pdf}
          \caption{%
            Die Abbildung zeigt die typische Wahl einer Basisfunktion $η_i$ über dem $i$.~Vertex des diskretisierten Berechnungsgebietes $\domain$.
            Aufgrund ihrer Form wird sie auch Hutfunktion genannt.
            Sie ist stückweise linear und identisch zu Null auf nicht benachbarten Dreiecken.%
          }
          \label{fig:hat-function}
        \end{figure}
      % paragraph schritt_3_wahl_der_basisfunktionen (end)

      \paragraph{Schritt 4: Konstruktion der algebraischen Formulierung} % (fold)
      \label{par:schritt_4_konstruktion_der_algebraischen_formulierung}
        \hfill\\
        Das Einsetzen der Diskretisierung in die ursprüngliche schwache Formulierung der partiellen Differentialgleichung führt zu einem System von algebraischen Gleichungen.
        Die in dieser Arbeit besprochenen Fälle resultieren in linearen Gleichungssystemen mit dünnbesetzten Matrizen.
        Diese hängen zudem stark von der gewählten Diskretisierung ab.
        \[
          \roundBrackets{ \integral{\domain}{}{ \scalarProduct{\nabla η_i}{\nabla η_j} }{λ} }_{i,j} =
          \begin{pmatrix}
            1 & -1 & & & \\
            -1 & 2 & -1 & & \\
            & -1 & 2 & -1 & \\
            & & -1 & 2 & -1 \\
            & & & -1 & 1
          \end{pmatrix}
        \]
        \[
          \roundBrackets{ \integral{\domain}{}{ \scalarProduct{\nabla η_i}{\nabla η_j} }{λ} }_{i,j\in I} =
          \begin{pmatrix}
            2 & -1 & \\
            -1 & 2 & -1 \\
            & -1 & 2 \\
          \end{pmatrix}
        \]
      % paragraph schritt_4_konstruktion_der_algebraischen_formulierung (end)

      \paragraph{Schritt 5: Lösung des algebraischen Systems} % (fold)
      \label{par:schritt_5_lsen_des_algebraischen_systems}
        \hfill\\
        Die erhaltenen algebraischen Gleichungen sind äquivalent zu der schwachen Formulierung des diskretisierten Problems.
        In den hier betrachteten Fällen existieren eindeutige Lösungen.
      % paragraph schritt_5_lösen_des_algebraischen_systems (end)
    % subsection finite_elemente_methode (end)

    \subsection{Sparse Matrix Methoden} % (fold)
    \label{sub:sparse_matrix_methoden}

    % subsection sparse_matrix_methoden (end)

    \subsection{Aufbau und Funktionsweise des Grafikprozessors} % (fold)
    \label{sub:aufbau_und_funktionsweise_des_grafikprozessors}
      Der Grafikprozessor (GPU, engl.: \textit{graphics processing unit}) eines Computers stellt eine weitere Prozessorart gegenüber dem Hauptprozessor (CPU, engl: \textit{central processing unit}) dar.
      Im Gegensatz zur CPU ist der Aufbau und die Funktionsweise der GPU auf konkrete Aufgaben spezialisiert.
      In ihrer einfachsten Form generiert die GPU zwei- und dreidimensionale Grafiken, Bilder und Videos, die grafische Benutzeroberflächen, Computerspiele, Video- und Bildbearbeitung ermöglichen.
      Die moderne GPU ist ein massiv paralleler Multiprozessor optimiert für visuelle Berechnungen (engl.: \textit{visual computing}).
      Um dem Benutzer eine visuelle Interaktion in Echtzeit zu bieten, besitzt die GPU eine vereinheitlichte Grafik- und Prozessorarchitektur (engl.: \textit{graphics and computing architecture}).
      Diese dient sowohl als programmierbarer Grafikprozessor sowie als skalierbare parallele Plattform.
      Auf herkömmlichen Systemen werden zudem CPUs mit GPUs verbunden, um ein sogenanntes heterogenes System zu bilden, welches die Vorteile der jeweiligen Prozessorarten ausnutzt.
      \cite[S.~A3]{Patterson2011}

      Gerade bei parallelisierbaren Problemen mit kohärenten und linearen Speicherzugriffen, die auf Matrix- und Vektoroperationen aufbauen, stellt die Verwendung der GPU eine enormen Anstieg der Effizienz dar.
      Aus Abschnitt \ref{sub:finite_elemente_methode} wird ersichtlich, dass gerade ein Programm, welches auf der Finite-Elemente-Methode basiert, durch die GPU stark beschleunigt werden könnte.
      Auch die Auflösung der Diskretisierung des zugrundeliegenden Berechnungsgebietes könnte erhöht werden, da die GPU mit der Anzahl der finiten Elemente skalieren kann.

      \subsubsection{GPU Computing mit CUDA} % (fold)
      \label{ssub:gpu_computing_mit_cuda}
        \enquote{GPU Computing} ist das Ausnutzen der GPU für Berechnungen mithilfe einer parallelen Programmiersprache und einer Programmierschnittstelle (API, engl.: \textit{application programming interface}).
        Das von NVIDIA entwickelte CUDA ist ein skalierbares paralleles Programmiermodell, eine parallele Programmiersprache und  eine API, die es dem Programmierer erlauben, die typische Grafikschnittstelle der GPU zu umgehen und diese direkt durch die Sprachen C und C++ zu programmieren \cite[S.~A5]{Patterson2011}.
        Das CUDA Programmiermodell arbeitet nach dem Prinzip \enquote{Single-Program Multiple Data} (SPMD).
        Der Programmierer schreibt hierbei ein Programm für einen einzelnen Thread, welches dann durch die GPU durch mehrere Threads instanziiert und ausgeführt wird \cite[S.~A5]{Patterson2011}.
      % subsubsection gpu_computing_mit_cuda (end)

      \subsubsection{Architektur einer modernen GPU} % (fold)
      \label{ssub:architektur_einer_modernen_gpu}
        Abbildung \ref{fig:gpu-architecture} zeigt die Architektur einer typischen CUDA-fähigen NVIDIA GPU.
        \begin{figure}[h]
          \center
          \includegraphics[width=0.95\textwidth]{images/gpu_architecture.png}
          \caption{%
            Die Abbildung zeigt eine schematische Darstellung der Architektur einer CUDA-fähigen GPU.
            \cite[S.~9]{Kirk2010}
          }
          \label{fig:gpu-architecture}
        \end{figure}
      % subsubsection architektur_einer_modernen_gpu (end)
    % subsection aufbau_und_funktionsweise_des_grafikprozessors (end)

  % section grundlagen (end)
\end{document}